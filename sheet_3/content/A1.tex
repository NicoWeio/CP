\section{Aufgabe 1: Lennard-Jones-Fluid}

\subsection{Theorie und Vorbereitung}
In dieser Aufgabe soll ein Lennard-Jones-Fluid simuliert werden.
Dazu wird ein Ensemble von $N=n^2$ Teilchen in einem Würfel mit Kantenlänge $L=2 n \sigma$ betrachtet.
Es gelten periodische Randbedingungen, sodass die Teilchen sich bildlich gesehen in einem Torus bewegen.
Die Teilchen wechselwirken über das Lennard-Jones-Potential
\begin{equation}
    \phi(r) = 4 \epsilon \left[ \left( \frac{\sigma}{r} \right)^{12} - \left( \frac{\sigma}{r} \right)^6 \right]
\end{equation}
mit $r$ als Abstand der Teilchen und den Parametern $\epsilon$ und $\sigma$.
Die Wechselwirkung ist dabei auf den Abstand $r < \frac{L}{2}$ beschränkt.
Die Gleichung wird in Einheiten von $\sigma$ und $\epsilon$ angegeben und die Masse der Teilchen ist $m=1$.
\\
Die Bewegung der Teilchen wird durch das Newtonsche Bewegungsgesetz
\begin{equation}
    m \ddot{\vec{r}} = \vec{F}(\vec{r}) = - \nabla \phi(\vec{r})
\end{equation}
beschrieben.
Die Kraft $\vec{F}$ bzw. Beschleunigung $\ddot{\vec{r}}$ (da $m=1$) ergibt sich aus dem negativen Gradienten des Potentials $\phi$.
In diesem Fall ist die Kraft also
\begin{equation}
    \vec{F}(\vec{r}) = 24 \epsilon \left[ 2 \left( \frac{\sigma}{r} \right)^{14} - \left( \frac{\sigma}{r} \right)^8 \right] \frac{\vec{r}}{r^2} \,.
\end{equation}
wobei zu beachten ist, dass ein zusätzlicher Faktor $\frac{1}{r}$ durch die Normierung des Basisvektor $\vec{e_r} = \frac{\vec{r}}{r}$ entsteht.
\\
Die Bewegungsgleichung wird mit dem Geschwindigkeits-Verlet-Algorithmus
\begin{align}
    \vec{r}_{i+1} &= \vec{r}_i + \vec{v}_i \mathrm{d}t + \frac{1}{2} \vec{a}_i \mathrm{d}t^2 \\
    \vec{v}_{i+1} &= \vec{v}_i + \frac{1}{2} \left( \vec{a}_{i+1} + \vec{a}_{i} \right) \mathrm{d}t
\end{align}
gelöst, dabei gibt $dt$ die Zeitschrittweite an.
Wenn $n$ die Anzahl an Schritten angibt, so wird für jeden betrachteten Zeitpunkt
\begin{equation*}
    t = i \cdot \mathrm{d}t \, ,  \qquad i = 0, 1, 2, \dots, n
\end{equation*}
die Position $\vec{r}_i$ und Geschwindigkeit $\vec{v}_i$ berechnet, bis die Messzeit $T = n \cdot \mathrm{d}t$ erreicht ist.
Der Verlet-Algorithmus ist symplektisch, d.h. er erhält die Energie des Systems.
\\
Die Anfangsbedingungen werden durch eine Gitteranordnung mit Abstand $\sigma$ und zufälligen Geschwindigkeiten gewählt.
Die Geschwindigkeiten werden zuerst um den Schwerpunkt zentriert
\begin{equation}
    \vec{v}_i \rightarrow \vec{v}_i - \vec{v}_S = \vec{v}_i - \frac{1}{N} \sum_{i=1}^N \vec{v}_i
\end{equation}
Allgemein kann aus der kinetische Energie des Gesamtsystems $E_{kin}$
\begin{equation}
    E_{kin} = \frac{1}{2} \sum_{i=1}^N m_i v_i^2
\end{equation}
die Temperatur $T$ über den Zusammenhang
\begin{equation}
    T = \frac{2}{N_f k_B} E_{kin}
    \label{eq:temp}
\end{equation}
berechnet werden, wobei $N_f = 3N-3$ die Freiheitsgrade des Systems angibt.
\\
Die Anfangsgeschwindigkeiten werden skaliert, sodass die kinetische Energie der Teilchen der Temperatur $T$ entspricht.
Aus Gleichung \eqref{eq:temp} folgt unmittelbar der Skalierungsfaktor
\begin{equation}
    \alpha(t) = \sqrt{\frac{T}{T(t)}} \,.
    \label{eq:isokin}
\end{equation}
welcher auch das isokinetische Thermostat definiert.
Bei Verwendung des isokinetischen Thermostats werden die Geschwindigkeiten in jedem Zeitschritt neu skaliert, sodass die Temperatur konstant bleibt.
Ohne Thermostat werden die Geschwindigkeiten nur bei der Initialisierung skaliert.

\subsection{Äquilibrierung und Messung}
Befor die Messung der Observablen $T, E_\text{kin}, E_\text{pot}, v_S, r_{ij}$ durchgeführt werden kann, muss das System zunächst äquilibriert werden.
Das System benötigt eine gewisse Zeit, um einen stationären und physikalisch sinnvollen Zustand zu erreichen.
Die beiden Funktionen \texttt{equilibrate} und \texttt{measure} in \texttt{A1.cpp} führen die Äquilibrierung bzw. Messung durch.
Sie unterscheiden sich zum einen in der Anzahl der Zeitschritte $n_\text{equi}$ bzw. $n_\text{meas}$ und zum anderen werden keine Werte gemessen/gespeichert, wenn die Funktion \texttt{equilibrate} aufgerufen wird.
\\
Die Simulation wird vorerst ohne Thermostat für eine Anfangstemperatur $T(0)=1$ für $n=4,8,16$ durchgeführt.
Es wird die Schrittweite $dt=0.01$ verwendet und eine Messung mit $n_\text{equi}=1000$ Schritten ohne Äquilibrierung gestartet.
So können wir die Äquilibrierungszeit abschätzen.
\\
In den Abbildungen \ref{fig:obs4}($n=4$), \ref{fig:obs8}($n=8$), \ref{fig:obs16}($n=16$) sind die Energien $E$ und die Temperatur $T$ in Abhängigkeit der Zeit $t$ dargestellt.
Das System ist äquilibriert, wenn die Energien und die Temperatur konstant sind.
Folgende Äquilibrierungszeit werden den Abbildungen entnommen:
\begin{align*}
    n=4: \quad & t_\text{equi} \approx 100 \\
    n=8: \quad & t_\text{equi} \approx 100 \\
    n=16: \quad & t_\text{equi} \approx 200
\end{align*}

\begin{figure}
    \centering
    \includegraphics[width=0.8\textwidth]{content/plots/observable_n4_T1_nothermo.pdf}
    \caption{Energien und Temperatur für $n=4$ und $T(0)=1$ ohne Thermostat.}
    \label{fig:obs4}
\end{figure}
\begin{figure}
    \centering
    \includegraphics[width=0.8\textwidth]{content/plots/observable_n8_T1_nothermo.pdf}
    \caption{Energien und Temperatur für $n=8$ und $T(0)=1$ ohne Thermostat.}
    \label{fig:obs8}
\end{figure}
\begin{figure}
    \centering
    \includegraphics[width=0.8\textwidth]{content/plots/observable_n16_T1_nothermo.pdf}
    \caption{Energien und Temperatur für $n=16$ und $T(0)=1$ ohne Thermostat.}
    \label{fig:obs16}
\end{figure}