\section{Aufgabe 2 – Der Wetterfrosch}
\label{sec:A2}
In dieser Aufgabe soll ein einfach Wettermodell untersucht werden.
Es wird das chaotische Lorenz-Modell, bestehend aus drei gekoppelten Differentialgleichungen
\begin{align}
    \dot{x} &= -\sigma x + \sigma y \\
    \dot{y} &= -x z + r x - y \\
    \dot{z} &= x y - b z
\end{align}
mit den Kooridnaten $x$, $y$ und $z$ betrachtet.
Es werden die Parameter $\sigma = 10$, $b = \frac{8}{3}$ und $r = 20$ bzw. $r = 28$ verwendet.
\\
Die Differentialgleichungen werden mit dem Runge-Kutta-Verfahren 4. Ordnung gelöst.
Die Schrittweite wird auf $\Delta t = 0.01$ gesetzt und die DGL bis $T = 100$ integriert.
\\
Die Anfangswerte werden auf $x_0 = 1$, $y_0 = 1$ und $z_0 = 1$ gesetzt.
Andere Anfangswerte führen zu leicht anderen Trajektorien mit anderen Attraktoren.
\\
Zuerst wird die \textbf{Projektion auf die $x$-$y$-Ebene} betrachtet.
In Abbildung \ref{fig:A2_lorenz_xy20} ist die Trajektorie für $r = 20$ und in Abbildung \ref{fig:A2_lorenz_xy28} für $r = 28$ dargestellt.
\begin{figure}
    \centering
    \includegraphics[width=1.0\textwidth]{code/build/A2_xy_r20.pdf}
    \caption{Projektion der Trajektorie auf die $x$-$y$-Ebene für $r = 20$.}
    \label{fig:A2_lorenz_xy20}
\end{figure}

\begin{figure}
    \centering
    \includegraphics[width=1.0\textwidth]{code/build/A2_xy_r28.pdf}
    \caption{Projektion der Trajektorie auf die $x$-$y$-Ebene für $r = 28$.}
    \label{fig:A2_lorenz_xy28}
\end{figure}
Im Fall $r=20$ ist die Trajektorie beschränkt und die Trajektorie ist eine geschlossene Kurve.
Für $r=28$ ist die Trajektorie nicht mehr beschränkt und die Trajektorie ist eine chaotische Kurve.
Andere Anfangsbedinungen können dazu führen, dass die Trajektorie zu einem anderen Attraktor divergiert.
\FloatBarrier

Als nächstes wird der \textbf{Poincaré-Schnitt} zu $z = 20 = const.$ und $\dot{z} < 0$ betrachtet.
In Abbildung \ref{fig:A2_lorenz_poincare20} ist der Poincaré-Schnitt für $r = 20$ und in Abbildung \ref{fig:A2_lorenz_poincare28} für $r = 28$ dargestellt.
\begin{figure}
    \centering
    \includegraphics[width=0.9\textwidth]{code/build/A2_poincare_r20.pdf}
    \caption{Poincaré-Schnitt für $r = 20$.}
    \label{fig:A2_lorenz_poincare20}
\end{figure}

\begin{figure}
    \centering
    \includegraphics[width=0.9\textwidth]{code/build/A2_poincare_r28.pdf}
    \caption{Poincaré-Schnitt für $r = 28$.}
    \label{fig:A2_lorenz_poincare28}
\end{figure}
\FloatBarrier

Zuletzt werden die Trajektorien im \textbf{3-dimensionalen Raum} betrachtet.
In Abbildung \ref{fig:A2_lorenz3d20} ist die Trajektorie für $r = 20$ und in Abbildung \ref{fig:A2_lorenz3d28} für $r = 28$ dargestellt.
\begin{figure}
    \centering
    \includegraphics[width=0.9\textwidth]{code/build/A2_3d_r20.pdf}
    \caption{Trajektorie im 3-dimensionalen Raum für $r = 20$.}
    \label{fig:A2_lorenz3d20}
\end{figure}

\begin{figure}
    \centering
    \includegraphics[width=0.9\textwidth]{code/build/A2_3d_r28.pdf}
    \caption{Trajektorie im 3-dimensionalen Raum für $r = 28$.}
    \label{fig:A2_lorenz3d28}
\end{figure}
\FloatBarrier

