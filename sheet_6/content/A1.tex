\section{Aufgabe 1 – Bifurkationsdiagramme}
\label{sec:A1}
\subsection{Logistische Abbildung}
Die logistische Abbildung ist definiert als
\begin{equation}
    x_{n+1} = r x_n (1 - x_n)
\end{equation}
mit $x_n \in [0, 1]$.
Es werden Anfangswerte $x_0$ zwischen $0$ und $1$ mit einer Schrittweite von $0.001$ und $r$ zwischen $0$ und $4$ mit einer Schrittweite von $0.01$ gewählt.
\\
Folgende Fälle können durch die Wahl des Parameters $r$ erzeugt werden
\begin{itemize}
    \item $r < 1$: Die Folge konvergiert gegen $0$
    \item $1 <= r < 3$: Die Folge konvergiert gegen einen konstanten Wert
    \item $3 <= r < 3.57$: Die Folge konvergiert gegen einen periodischen Wert
    \item $r > 3.57$: Chaos
\end{itemize}

\begin{figure}
    \centering
    \includegraphics[width=0.9\textwidth]{code/build/A1_bifurkation_log.pdf}
    \caption{Bifurkationsdiagramm für die logistische Abbildung.}
    \label{fig:A1_bifurkation}
\end{figure}
\FloatBarrier

\subsection{Kubische Abbildung}
Die kubische Abbildung ist definiert als
\begin{equation}
    x_{n+1} = r x_n^3 - x_n
\end{equation}
mit $x_n \in [-\sqrt{1+r}, \sqrt{1+r}]$.
\begin{figure}
    \centering
    \includegraphics[width=0.9\textwidth]{code/build/A1_bifurkation_cube.pdf}
    \caption{Bifurkationsdiagramm für die kubische Abbildung.}
    \label{fig:A1_kubisch}
\end{figure}
Es werden die gleichen Anfangswerte x0 verwendet.
Die Grenzen für $r$ werden auf $[0, 3]$ gesetzt.
Für größere Werte von $r$ divergiert die Folge.