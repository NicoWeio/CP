\section{Aufgabe 2 –  Teilchenschwarm-Optimierung}
\label{sec:A2}

Zur Lösung des mathematischen Problems \textit{Minimierung mehrdimensionaler Funktionen} wird ein physikalischer Ansatz verwendet.
Dabei wird ein Schwarm aus $N=10$ Teilchen simuliert, die sich in einem mehrdimensionalen Raum bewegen.
Die Orte $\vec{r}_{n+1}$ und Geschwindigkeiten $\vec{v}_{n+1}$ der Teilchen werden dabei iterativ nach folgenden Gleichungen berechnet:
\begin{align}
    \vec{r}_{n+1} &= \vec{r}_n + \vec{v}_{n} \\
    \vec{v}_{n+1} &= \omega \vec{v}_n + c_1 r_1 (\vec{r}_n^\text{pers. best} - \vec{r}_n) + c_2 r_2 (\vec{r}^\text{global best} - \vec{r}_n)
\end{align}
Dabei sind $\omega=0.8$, $c_1=0.1$ und $c_2=0.1$ Gewichtungsfaktoren und $r_1, r_2 \in [0,1]$ Zufallszahlen, die in jeder Iteration neu gezogen werden.
Die persönlich beste Position $\vec{r}_n^\text{pers. best}$ wird für jedes Teilchen gespeichert und die global beste Position $\vec{r}^\text{global best}$ wird für den gesamten Schwarm gespeichert.
Als beste Position wird diejenige gewählt, die den kleinsten Funktionswert $f(\vec{r})$ liefert.
\\
Es soll das globale Minimum der Funktion
\begin{align}
   f(\vec{r}) = f (x, y) = (x - 1.9)^2 + (y - 2.1)^2 + 2 \cos(4x + 2.8) + 3 \sin(2y + 0.6)
\end{align}
gefunden werden.
\\
Die Startpositionen der Teilchen werden zufällig im Bereich $x, y \in [-5, 5]$ gewählt.
Die Startgeschwindigkeiten werden zufällig im Bereich $v_x, v_y \in [-1, 1]$ gewählt.
\\
Der Algorithmus wird mit $N=10$ Teilchen und $M=100$ Iterationen ausgeführt.
\\
In Abbildung \ref{fig:A2_b} ist die Entwicklung der besten Teilchenpositionen in Abhängigkeit der Iterationen dargestellt.
Es werden nur die ersten 20 Iterationen dargestellt, da sich die Positionen der Teilchen danach nicht mehr wesentlich ändern.
\begin{figure}
    \centering
    \includegraphics[width=0.9\textwidth]{code/build/A2_b.pdf}
    \caption{Verlauf der besten persönlichen Funktionswerte $f(\vec{r}_n^\text{pers. best})$ für jedes Teilchen, sowie des besten globalen Funktionswerts $f(\vec{r}^\text{global best})$ für den gesamten Schwarm in Abhängigkeit der Iterationen.}
    \label{fig:A2_b}
\end{figure}
Zuletzt wird der Verlauf der Teilchenpositionen in einer 3D-Animation \texttt{A2\_c\_w08.mp4} für $\omega=0.8$ dargestellt.
In Abbildung \ref{fig:A2_c} ist ein Schnappschuss aus der Animation zusehen.
Dieser zeigt die Spur der Teilchenpositionen, sowie die Funktion $f(x, y)$.
\\
Die beste Position des Schwarmes ist
%x = 1.67055, y = 2.06245, f(x,y) = -4.94241
\begin{align}
    \vec{r}^\text{global best} &= \begin{pmatrix}
        1.67055 \\
        2.06245
    \end{pmatrix}\\
    f(\vec{r}^\text{global best}) &= -4.94241
\end{align}
und stellt somit das globale Minimum dar.
\begin{figure}
    \centering
    \includegraphics[width=1.0\textwidth]{code/build/A2_c.pdf}
    \caption{Spur der Teilchenpositionen, sowie die Funktion $f(x, y)$.}
    \label{fig:A2_c}
\end{figure}