\section{Aufgabe 1 – Eindimensionale Optimierung}
\label{sec:A1}
Im Folgenden werden zwei Verfahren
\begin{itemize}
    \item das \textit{Intervallhalbierungs-Verfahren} und
    \item das \textit{Newton-Verfahren}
\end{itemize}
zur Minimierung einer eindimensionalen Funktion $f(x)$ vorgestellt.
\\
Beispielhaft wird die Funktion
\begin{equation}
  f(x) = x^2 - 2
\end{equation}
untersucht, wobei das Minimum bei $x=0$ liegt.
Die Verfahren werden abegbrochen, wenn die Differenz zwischen zwei aufeinanderfolgenden Iterationen kleiner als $x_e = 10^{-9}$ ist.
Die Implementierung beider Methoden ist in \texttt{optimize.h} zu finden.
\\
In Abbildung \ref{fig:A1_values} sind die Verläufe der Parameter $x, y, z$ des Bisektion-Verfahrens und $x$ des Newton-Verfahrens in Abhängigkeit der Iterationen dargestellt.
Das Bisektion-Verfahren benötigt 61 Iterationen, um das Minimum zu finden, während das Newton-Verfahren bereits nach 3 Iterationen konvergiert.
\begin{figure}
    \centering
    \includegraphics[width=1.0\textwidth]{code/build/A1_values.pdf}
    \caption{Verlauf der Parameter $x, y, z$ des Bisektion-Verfahrens und $x$ des Newton-Verfahrens in Abhängigkeit der Iterationen.}
    \label{fig:A1_values}
\end{figure}

\begin{figure}
    \centering
    \includegraphics[width=1.0\textwidth]{code/build/A1_error.pdf}
    \caption{Verlauf der Differenz zwischen zwei aufeinanderfolgenden Iterationen in Abhängigkeit der Iterationen.}
    \label{fig:A1_error}
\end{figure}
\FloatBarrier