\section{Aufgabe 2 – Oszillierender Rechtseckmembran}
\subsection{a) Diskretisierung der Wellengleichung}
Vorerst soll die Wellengleichung in zwei Dimensionen
\begin{equation}
    \label{eqn:wave2d}
    \frac{\partial^2 u}{\partial t^2} = c^2 \left( \frac{\partial^2 u}{\partial x^2} + \frac{\partial^2 u}{\partial y^2} \right)
\end{equation}
diskretisiert werden.
Der Raum ($x, y$), sowie die Zeit ($t$) werden in äquidistante Intervalle unterteilt:
\begin{align}
    t_n &= n \Delta t \\
    x_i &= i \Delta x \\
    y_j &= j \Delta y
\end{align}
Die kontinuierliche Funktion $u(x, y, t)$ wird durch die diskrete Funktion $u_{i,j}^n$ ersetzt.
Sie gibt die Auslenkung des Membrans an.
\\
Die diskrete Form der zweiten Ableitung nach dem Ort $x$ ist
\begin{equation}
    \frac{\partial^2 u}{\partial x^2} \approx \frac{u_{i+1,j}^n - 2 u_{i,j}^n + u_{i-1,j}^n}{\Delta x ^2}
\end{equation}
und analog für die Ableitung nach $y$ und $t$.
\\
Die diskreten Ableitungen werden in die Wellengleichung eingesetzt
\begin{equation*}
    \frac{u_{i,j}^{n+1} - 2 u_{i,j}^n + u_{i,j}^{n-1}}{\Delta t^2} = c^2 \left( \frac{u_{i+1,j}^n - 2 u_{i,j}^n + u_{i-1,j}^n}{\Delta x ^2} + \frac{u_{i,j+1}^n - 2 u_{i,j}^n + u_{i,j-1}^n}{\Delta y ^2} \right)
\end{equation*}
und diese anschließend nach $u_{i,j}^{n+1}$ umgestellt
\begin{equation}
    u_{i,j}^{n+1} = 2 u_{i,j}^n - u_{i,j}^{n-1} + \frac{c^2 \Delta t^2}{\Delta x^2} \left( u_{i+1,j}^n - 2 u_{i,j}^n + u_{i-1,j}^n \right) + \frac{c^2 \Delta t^2}{\Delta y^2} \left( u_{i,j+1}^n - 2 u_{i,j}^n + u_{i,j-1}^n \right)
    \label{eqn:discretewave}
\end{equation}
, um die Auslenkung der Membran zum Zeitpunkt $t_{n+1}$ zu berechnen.

\subsection{b) Stabilitätsanalyse}
Jede Störung der Lösung $\delta u(x,y,t)$ kann in ihre Fourierkomponenten
\begin{equation}
    \delta u(x,y,t) = \tilde{c}_{\omega,k_x,k_y} \exp \left( i \left( k_x x + k_y y - \omega t \right) \right)
\end{equation}
zerlegt werden.
Nach Einsetzen in \autoref{eqn:discretewave} und Verwendung der exponentiellen Darstellung von $\cos(x) = \frac{1}{2} \left( e^{ix} + e^{-ix} \right)$ ergibt sich
\begin{equation*}
    cos(\omega t) - 1 = \frac{c^2 \Delta t^2}{\Delta x^2} (\cos(k_x x) - 1) + \frac{c^2 \Delta t^2}{\Delta y^2} (\cos(k_y y) - 1) \, .
\end{equation*}
Uns interessiert nur der Betrag der Gleichung, da dieser die Stabilität der Lösung angibt.
Unter Verwendung der Dreiecksungleichung $|a+b| \leq |a| + |b|$ folgt
\begin{equation*}
    |cos(\omega t) - 1| \leq \frac{c^2 \Delta t^2}{\Delta x^2} |(\cos(k_x x) - 1)| + \frac{c^2 \Delta t^2}{\Delta y^2} |(\cos(k_y y) - 1)| \, .
\end{equation*}
Da $max(|\cos(K_x x) - 1|) = 2$ gilt (analog für $y$), ergibt sich
\begin{equation*}
    |cos(\omega t) - 1| \leq 2 \frac{c^2 \Delta t^2}{\Delta x^2} + 2 \frac{c^2 \Delta t^2}{\Delta y^2} \, .
\end{equation*}
Die linke Seite kann für $\omega \in \mathbb{R}$ maximal den Wert $2$ annehmen.
Daraus folgt für eine reelle Kreisfrequenz $\omega$ die Stabilitätsbedingung
\begin{equation}
    \Delta t \leq \frac{1}{c} \frac{\Delta x \Delta y}{\sqrt{\Delta x^2 + \Delta y^2}} \, .
    \label{eqn:stability}
\end{equation}
Für komplexe $\omega$ kann der Betrag von $\cos(\omega t) - 1$ größer als $2$ werden und die Stabilitätsbedingung ist nicht mehr erfüllt.

\subsection{c) Anfangsbedinungen}
Die Anfangsbedingungen der Wellengleichung sind typischerweise die Auslenkung $u(x,y,0)$ und die Geschwindigkeit $\frac{\partial u}{\partial t}(x,y,0)$.
Dabei entspricht $u(x,y,0)$ der diskreten Funktion $u_{i,j}^0$.
\\
Zur Verwendung der zweiten Bedingungen wird die Wellengleichung für $n=0$ betrachtet
\begin{equation}
    u_{i,j}^1 = 2 u_{i,j}^0 - u_{i,j}^{-1} + \frac{c^2 \Delta t^2}{\Delta x^2} \left( u_{i+1,j}^0 - 2 u_{i,j}^0 + u_{i-1,j}^0 \right) + \frac{c^2 \Delta t^2}{\Delta y^2} \left( u_{i,j+1}^0 - 2 u_{i,j}^0 + u_{i,j-1}^0 \right) \, .
\end{equation}
Der Funktionswert $u_{i,j}^{-1}$ ist nicht bekannt, daher wird dieser Term durch die Ableitung nach der Zeit ersetzt und nach $u_{i,j}^{-1}$ umgestellt:
\begin{align}
    \frac{\partial u}{\partial t}(x,y,0) &= \frac{u_{i,j}^1 - u_{i,j}^{-1}}{2 \Delta t}\\
    u_{i,j}^{-1} &= u_{i,j}^1 - 2 \Delta t \frac{\partial u}{\partial t}(x,y,0)
\end{align}
Somit kann die Anfangsbedingung der Geschwindigkeit zur Bestimmung der Auslenkung $u_{i,j}^{1}$ verwendet werden.

\subsection{d) Entdimensionalisierung}
Die Wellengleichung \autoref{eqn:wave2d} wird in eine dimensionslose Form gebracht.
Dazu werden die Variablen $x$, $y$ und $t$ durch die dimensionslosen Variablen
\begin{align*}
    \tilde{x} &= \frac{x}{\alpha} \\
    \tilde{y} &= \frac{y}{\alpha} \\
    \tilde{t} &= \frac{t}{\tau}
\end{align*}
ausgedrückt.
Unter Verwendung der Kettenregel (exemplarisch für $x$)
\begin{equation}
    \frac{\partial u}{\partial x} = \frac{\partial u}{\partial \tilde{x}} \frac{\partial \tilde{x}}{\partial x} = \frac{1}{\alpha} \frac{\partial u}{\partial \tilde{x}}
\end{equation}
werden die dimensionslosen Variablen in die Wellengleichung eingesetzt:
\begin{equation}
    \frac{\partial^2 u}{\partial \tilde{t}^2} = \frac{\tau ^2 c^2}{\alpha^2} \frac{\partial^2 u}{\partial \tilde{x}^2} + \frac{\partial^2 u}{\partial \tilde{y}^2} \, .
\end{equation}
Es folgt die dimensionslose Wellengleichung
\begin{equation}
    \frac{\partial^2 u}{\partial \tilde{t}^2} = \frac{\partial^2 u}{\partial \tilde{x}^2} + \frac{\partial^2 u}{\partial \tilde{y}^2}
    \label{eqn:wave2d_dimless}
\end{equation}
für $\tau = \frac{\alpha}{c}$.

\subsection{e) Simulation}
Die Simulation wird für $\Delta x = \Delta y = 0.01$ für die gegebenen Anfangsbedingungen durchgeführt.
Es werden verschiedene Zeitschritte $\Delta t$ verwendet, um die Stabilitätsbedingung \autoref{eqn:stability} zu überprüfen.
Die Ergebnisse können in den Animationen (z.B. \texttt{A2\_dtE-2.mp4} für $\Delta t = 0.01$) betrachtet werden.
\\
Für $\Delta t = 0.01$ ist die Simulation nicht stabil.
Die theoretische Grenze wird durch die Stabilitätsbedingung \autoref{eqn:stability} angegeben und liegt bei $\Delta t = 0.00707$.