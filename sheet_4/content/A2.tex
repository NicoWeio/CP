\section{Aufgabe 2 – Diffusionsgleichung}
\label{sec:A2}
Zur Lösung der eindimensionalen Diffusionsgleichung
\begin{equation}
    \frac{\partial u}{\partial t} = D \frac{\partial^2 u}{\partial x^2}
\end{equation}
mit Konzentration $u(x,t)$ und Diffusionskoeffizient $D=1$ wird das FTCS (Forward-Time-Central-Space) Verfahren verwendet.
Die Gleichung wird dabei in die diskrete Form gebracht:
\begin{equation}
    \frac{u_{i}^{n+1}-u_{i}^{n}}{\Delta t} = D \frac{u_{i+1}^{n}-2u_{i}^{n}+u_{i-1}^{n}}{\Delta x^2}
\end{equation}
und nach $u_{i}^{n+1}$ umgestellt:
\begin{equation}
    u_{i}^{n+1} = u_{i}^{n} + \frac{D \Delta t}{\Delta x^2} (u_{i+1}^{n}-2u_{i}^{n}+u_{i-1}^{n})
\end{equation}
Das FTCS Verfahren ist in der Methode \text0.00001$\Delta t$ wird abhängig von den Anfangsbedingungen gewählt.

\subsection{Konstante Anfangsbedingung}
Vorerst wird eine konstante Konzentration $u(x,0)=a=1$ angenommen.
Diese Anfangsbedinung kann mit der Methode \texttt{void set\_initial\_const(double a)} gesetzt werden.
Die Konzentration $u(x,t)$ bleibt dabei unverändert, wie in der Animation \texttt{A2\_a).mp4} mit $\Delta t = 0.00001$ und $T = 0.01$ zu sehen ist.

\subsection{Stabilitätsbedingung mit Delta-Peak}
Als nächstes wird eine Delta-Peak mit Anfangsbedingung $u(x,0)=\delta(x-x_0)$ mit $x_0=0.5$ verwendet.
Die Höhe des Peaks wird durch die Variable $a$ beschrieben und wird auf $a=1$ gesetzt.
Diese Anfangsbedinung kann mit der Methode \texttt{void set\_initial\_delta(double x0, double a)} gesetzt werden.
\\
Die Stabilitätsbedingung für das FTCS Verfahren ist:
\begin{equation}
    \Delta t \leq \frac{\Delta x^2}{2D}
\end{equation}
Für $\Delta x = 0.01$ ergibt sich $\Delta t \leq 5 \cdot 10^{-4}$.
\\
Zuerst wird eine Schrittweite $\Delta t = 10^{-5}$ gewählt, welche die Stabilitätsbedingung erfüllt.
Die Animation \texttt{A2\_b)slow.mp4} zeigt die Konzentration $u(x,t)$ für $T=0.01$.
Der Peak breitet sich dabei mit der Zeit aus bis eine konstante Konzentration erreicht wird.
\\
Wird die Schrittweite $\Delta t = 10^{-4}$ gewählt, so ist die Stabilitätsbedingung nicht mehr erfüllt.
Die Animation \texttt{A2\_b)fast.mp4} zeigt die Konzentration $u(x,t)$ für $T\approx 0.001$.
Der Peak läuft dabei aus dem Intervall $[0,1]$ heraus und die Konzentration ist nicht mehr erhalten.

\subsection{Lösung verschiedener Anfangsbedingungen}
Es soll die zeitabhängige Lösung drei verschiedener Anfangsbedingungen
\begin{align}
    u_1(x,0) &= \delta(x-0.5) && \quad \text{Delta-Peak}\\
    u_2(x,0) &= \theta(x-0.5) && \quad \text{Heaviside}\\
    u_3(x,0) &= \frac{1}{9} \sum_{n=1}^{9} \delta(x-0.1n) && \quad \text{Dirac ridge}
\end{align}
berechnet werden.
Die Anfangsbedingung können mit den Methoden \\
\texttt{void set\_initial\_delta(double x0, double a)}, \\
\texttt{void set\_initial\_heaviside(double x0, double a)}, \\
\texttt{void set\_initial\_ridge(double a)} \\
gesetzt werden.
\\
Folgende Zeiten $T$ mit Schrittweite $\Delta t$ werden dabei betrachtet:
\begin{align*}
    u_1(x,t): \qquad T_1 &= 0.1, && \quad \Delta t = 10^{-5}\\
    u_2(x,t): \qquad T_2 &= 1.0, && \quad \Delta t = 10^{-5}\\
    u_3(x,t): \qquad T_3 &= 0.02, && \quad \Delta t = 10^{-6}
\end{align*}

Die Animationen \texttt{A2\_c)u1.mp4}, \texttt{A2\_c)u2.mp4} und \texttt{A2\_c)u3.mp4} zeigen die Konzentration $u(x,t)$ für die drei verschiedenen Anfangsbedingungen.
Die gleichnamigen csv-Dateien enthalten die für die Animationen verwendeten Daten, welche die zeitabhängige Lösung darstellen.
\\
Ein Gleichgewichtszustand stellt sich ein, wenn die Kontentration an jedem Ort gleich ist.
Dieser Zustand stellt sich für die Anfangsbedingung $u_3(x,0)$ signifikant schneller ein als für die anderen beiden Anfangsbedingungen.
Dies liegt daran, dass die Anfangsbedingung $u_3(x,0)$ aus mehreren Delta-Peaks besteht, welche sich unabhängig voneinander ausbreiten.
\\
Das Volumenintegral (hier 1D) der Konzentration $u(x,t)$ ist zu jedem Zeitpunkt $t$ konstant.
Dies wird durch Reflexion an der Wand sichergestellt und kann in den Animationen beobachtet werden.
