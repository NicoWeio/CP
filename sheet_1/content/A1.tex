\FloatBarrier
\section{Aufgabe 1}
\label{sec:A1}
\subsection{a)}
\begin{figure}
    \centering
    \includegraphics{code/build/plot.pdf}
    \caption{Die erste Ableitung an $x=1,5$ in Abhängigkeit von der Schrittweite $h$ (halb-logarithmisch).
    Die kleinsten $h$ sind nicht verwendbar, da dort die Werte $x$ und $x+h$ nicht unterscheidbar sind.
    Die geringste Abweichung von dem analytischen Wert hat $h=0,01$ (orange).}
\end{figure}

\begin{figure}
    \centering
    \includegraphics{code/build/A1_a_x.pdf}
    \caption{Die Abweichung der ersten Ableitung mit $h=0,01$. Die Abweichungen sind proportinal zu $cos^2(x)$.
    Das heißt die Abweichungen sind größer je steiler die Steigungen sind.}
\end{figure}
\FloatBarrier
\FloatBarrier
\subsection{b)}

\begin{equation}
    f''_{zweipunkt}= f'_{zweipunkt}( x+h,h_{f'}) - f'_{zweipunkt}( x-h,h_{f'}) / (2h)
\end{equation}

\begin{figure}
    \centering
    \includegraphics{code/build/A1_b_h.pdf}
    \caption{Die zweite Ableitung an $x=1,5$ mit $h_{f'}=0,01$ in Abhängigkeit von der Schrittweite $h$ (halb-logarithmisch).
    Die kleinsten $h$ sind nicht verwendbar, da dort die Werte $x$ und $x+h$ nicht unterscheidbar sind.
    Die geringste Abweichung von dem analytischen Wert hat $h=0,01$ (orange).}
\end{figure}

\begin{figure}
    \centering
    \includegraphics{code/build/A1_b_x.pdf}
    \caption{Die Abweichung der zweiten Ableitung mit $h=0,01$, $h_{f'}=0,01$. 
    Im Gegensatz zu den Abweichungen der ersten Ableitung ist kein Muster zu erkennen.}
\end{figure}
\FloatBarrier
\FloatBarrier

\subsection{c)}
\begin{figure}
    \centering
    \includegraphics{code/build/A1_c_h.pdf}
    \caption{Die erste Ableitung an $x=1,5$ mit in Abhängigkeit von der Schrittweite $h$ (halb-logarithmisch).
    Die geringste Abweichung von dem analytischen Wert hat $h=0,01$ (orange).}
\end{figure}

\begin{figure}
    \centering
    \includegraphics{code/build/A1_c_x.pdf}
    \caption{Die Abweichung der ersten Ableitung mit $h=0,01$. 
    Im Gegensatz zu den Abweichungen der mit dem Zweipunktverfahren ist kein Muster zu erkennen.
    Die Abweichungen sind um $10^1$ kleiner.}
\end{figure}
\FloatBarrier
\FloatBarrier
\subsection{d)}
\begin{equation}
    f'_2= 
    \begin{cases}
         + \sin(x \text{mod} \pi) \quad x \in [0,\pi)\\
         - \sin(x \text{mod}  \pi) \quad x \in (-\pi, 0) \\
    \end{cases}
\end{equation}

\begin{figure}
    \centering
    \includegraphics{code/build/A1_d_h.pdf}
    \caption{Die erste Ableitung an $x=1,5$ mit in Abhängigkeit von der Schrittweite $h$ (halb-logarithmisch).
    Die geringste Abweichung von dem analytischen Wert hat $h_{zwei}=0,0001$ für das Zweipunktverfahren und $h_{vier}=0,01$ für das Vierpunktverfahren (orange).}
\end{figure}

\begin{figure}
    \centering
    \includegraphics{code/build/A1_d_x.pdf}
    \caption{Die ersten Ableitungen beider Methoden geben ähnliche Werte.
    Die Abweichungen sind für $x>0$ sehr klein
    und zwischen $-\pi$ und $0$ parabelförmig.}
\end{figure}

