\section{Aufgabe 3} \label{sec:A3}
\subsection*{a)} \label{sec:A3:a}
\begin{tcolorbox}
  Implementieren Sie sowohl das Euler-Verfahren, als auch das symmetrische Euler-Verfahren
  und starten Sie mit Anfangswerten $y_0 = 1$ und für das symmetrische Euler-Verfahren zusätzlich mit $y_1 = \exp(−∆t)$.
  Vergleichen Sie Ihre Ergebnisse mit der analytischen Lösung im Intervall $t ∈ [0, 10]$.
\end{tcolorbox}

\begin{figure}[H]
  \centering
  \includegraphics[width=\textwidth]{code/build/a3_a.pdf}
\end{figure}
\begin{figure}[H]
  \centering
  \includegraphics[width=\textwidth]{code/build/a3_a_diff.pdf}
\end{figure}
In der Abbildung mit den Differenzen zu den analytischen Werten
kann man erkennen, dass bei dem Euler-Verfahren die Differenz kleiner wird,
je mehr Punkte berechnet werden.
Bei dem symmetrischen Euler-Verfahren nimmt die Diskrepanz zu und osziliert
von negativen zu positiven Abweichungen.


\subsection*{b)} \label{sec:A3:b}
\begin{tcolorbox}
  Vergleichen Sie Ihre Ergebnisse aus dem vorigen Aufgabenteil mit den Ergebnissen,
  wenn Sie das Euler-Verfahren mit $y_0 = 1 − ∆t$
  und das symmetrische Euler-Verfahren mit $y_0 = 1$ und $y_1 = y_0 − ∆t$ starten.
  Deuten Sie Ihre Ergebnisse.
\end{tcolorbox}

\begin{figure}[H]
  \centering
  \includegraphics[width=\textwidth]{code/build/a3_b.pdf}
\end{figure}
\begin{figure}[H]
  \centering
  \includegraphics[width=\textwidth]{code/build/a3_b_diff.pdf}
\end{figure}
Vergleicht man die Abweichung mit Teil \hyperref[sec:A3:a]{a)}, sind diese größer.
Das heißt: Wie gut das Verfahren für das Problem ist, ist abhängig von den Startbedingungen.
