\section{Aufgabe 1 – Matrix Diagonalisierung mittels Potenzmethode}
\label{sec:A1}
%A = 1, -2, -3, 4,
%-2, 2, -1, 7,
%-3, -1, 3, 6,
% 4, 7, 6, 4;
Es sollen die Eigenwerte der Matrix $A$
\begin{equation}
    A = \begin{pmatrix}
    1 & -2 & -3 & 4 \\
    -2 & 2 & -1 & 7 \\
    -3 & -1 & 3 & 6 \\
    4 & 7 & 6 & 4
    \end{pmatrix}
\end{equation}
bestimmt werden.
Die Diagonalmatrix ergibt durch die Eigenwerte $\lambda_i$ auf der Diagonalen.
\\
Vorerst werden die Eigenwerte mithilfe der Bibliothek \textit{Eigen::eigenvalues()} auf
\begin{equation}
    \lambda_1 = -9.27794 \quad \lambda_2 = 4.59705 \quad \lambda_3 = 2.59583 \quad \lambda_4 = 12.0851
    \label{eq:eigenvalues}
\end{equation}
bestimmt.
\\
Die Potenzmethode für eine $4 \times 4$-Matrix ist in der Klasse \textit{EVpower} implementiert.
Dabei wird die Matrix $A$ mit einem Startvektor $v_0$ multipliziert und das Ergebnis $v_1$ normiert.
Dieser Vorgang wird $N=100$ mal wiederholt.
Der Vektor $v_n$ konvergiert gegen den Eigenvektor $v_n$ des betragsmäßig größten Eigenwertes $\lambda$.
\\
Der Eigenwert $\lambda$ berechnet sich dann nach
\begin{equation}
    \lambda = v_n^T A v_n
\end{equation}
, dabei ist $v_n$ normiert.
Anschließend wird die Matrix $A$ um den Eigenwert $\lambda$ und den Eigenvektor $v_n$ reduziert:
\begin{equation}
    A' = A - \lambda v_n v_n^T.
\end{equation}
Der Eigenwert wird somit auf Null gesetzt und es kann erneut die Potenzmethode angewendet werden um den nächsten betragsmäßig größten Eigenwert zu bestimmen.
\\
Es wird der Startvektor $v_0 = (1, 1, 1, 1)^T$ gewählt.
Die berechneten Eigenwerte stimmen auf mindestens $5$ Nachkommastellen mit den Eigenwerten aus Gleichung \eqref{eq:eigenvalues} überein.
Die Eigenwerte, sowie -vektoren werden beim Ausführen der Datei \textit{A1.cpp} in der Konsole ausgegeben.